%%******************************************************************************
%%
%% introduction.tex
%%
%%******************************************************************************
%%
%% Title......: Introduction
%%
%% Author.....: GSCAR-DFKI
%%
%% Started....: Nov 2013
%%
%% Emails.....: alcantara@poli.ufrj.br elael@poli.ufrj renan028@gmail.com
%%
%% Address....: Universidade Federal do Rio de Janeiro
%%              Caixa Postal 68.504, CEP: 21.945-970
%%              Rio de Janeiro, RJ - Brasil.
%%
%%******************************************************************************


%%******************************************************************************
%% SECTION - Eletronica
%%******************************************************************************

\section{Proposta 3 – PC embarcado e PC na base}

A terceira solução consiste na compra de uma eletrônica embarcada e um
computador em Pelican Case. Esta solução, apesar de ser a mais custosa, é mais
robusta devido ao grau de proteção elétrica e mecânica, além de já estar em
conformidade quanto aos requisitos de projeto.

\subsubsection{Arquitetura de Software Proposta 3}
Esta proposta é um meio termo entre a proposta com todo o processamento em terra
(proposta 1) e a proposta anterior, na qual todo o processamento de dados se dá
na eletrônica embarcada. Os componentes de software são basicamente os mesmos,
excetuando-se a ordem em que o processamento e a comunicação ocorrem.

Os dados dos sensores das operações de inserção e remoção de Stoplogs
(encoders, inclinômetro, sensores indutivos e sensor de pressão), assim como os
dados brutos provenientes do sonar e do módulo PanTilt serão conformados em um
computador embarcado e enviados via ethernet para um computador em superfície.

Já na superfície o componente Sonar-PanTilt realizará a fusão dos dados
referentes ao sonar e o módulo de reconstrução 3D irá gerar a imagem,
analogamente as outras propostas. Por fim os componentes de Monitoração e
Visualização enviarão os dados processados para o tablet. O componente de
segurança também está implementado no computador em terra.

A maior vantagem dessa arquitetura, na ótica de software, é a facilidade de
manutenção, uma vez que para a realização de qualquer alteração do software do
sistema não é necessário a abertura da envólucro do sistema embarcado. Já a sua
desvantagem seria a maior quantidade de trabalho para a programação de dois
sistemas.